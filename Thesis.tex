%%%%%%%%%%%%%%%%%%%%%%%%%%%%%%%%%%%%%%%%%%%%%%%%%%%%%%%%%%%%%%%%%%%%%
%% This is a (brief) model paper using the achemso class
%% The document class accepts keyval options, which should include
%% the target journal and optionally the manuscript type.
%%%%%%%%%%%%%%%%%%%%%%%%%%%%%%%%%%%%%%%%%%%%%%%%%%%%%%%%%%%%%%%%%%%%%
% \documentclass[journal=jctcce,manuscript=article]{achemso}
\documentclass[a4paper, 12pt, openany]{book}



% \setkeys{acs}{articletitle=false}

%%%%%%%%%%%%%%%%%%%%%%%%%%%%%%%%%%%%%%%%%%%%%%%%%%%%%%%%%%%%%%%%%%%%%
%% Place any additional packages needed here.  Only include packages
%% which are essential, to avoid problems later. Do NOT use any
%% packages which require e-TeX (for example etoolbox): the e-TeX
%% extensions are not currently available on the ACS conversion
%% servers.
%%%%%%%%%%%%%%%%%%%%%%%%%%%%%%%%%%%%%%%%%%%%%%%%%%%%%%%%%%%%%%%%%%%%%
\usepackage[version=3]{mhchem} % Formula subscripts using \ce{}
\usepackage{chemnum}
\usepackage{multirow}
\usepackage{textgreek}
\usepackage{textcomp}
\usepackage{setspace}
\usepackage{afterpage}
\usepackage{caption}
\usepackage{floatrow}
\usepackage{gensymb}
\usepackage{mathtools}
	
\usepackage[bottom=2cm,top=2cm, right=2cm, left=2cm]{geometry}


\usepackage{csquotes}
\usepackage{setspace}
\doublespacing
\usepackage{epigraph}
\renewcommand{\epigraphsize}{\small}
\renewcommand{\textflush}{flushright} \renewcommand{\sourceflush}{flushright}

\let\originalepigraph\epigraph 
\renewcommand\epigraph[2]{\originalepigraph{\textit{#1}}{\textsc{#2}}}
\setlength{\epigraphwidth}{0.5\textwidth}

\renewcommand{\textflush}{flushright} \renewcommand{\sourceflush}{flushright}




\usepackage[most]{tcolorbox}


\captionsetup{justification=raggedright,singlelinecheck=false}
\usepackage[labelfont=bf]{caption}
\usepackage{makecell}

\usepackage{dblfloatfix}
\usepackage{framed}

\usepackage{natbib}
\usepackage[acroymn]{glossaries}
\usepackage{tabularx} % for 'tabularx' environment
\usepackage{array}

\usepackage{acro}

\usepackage{tocloft}
\cftsetrmarg{7em}% Default is 2.55em

% \acsetup{hyperref=true}

\usepackage{enumitem}

\usepackage{multicol}
\usepackage{graphicx}
\usepackage{subcaption}

% glossary
\usepackage{hyperref}


\hypersetup{
    colorlinks,
    citecolor=black,
    filecolor=black,
    linkcolor=black,
    urlcolor=black
}

\usepackage{fancyhdr}
\pagestyle{fancy}

\renewcommand{\headrulewidth}{.01em} % Headline 
\renewcommand{\chaptermark}[1]{\markboth{\thechapter{\hspace{0.5cm}}\ #1}{}}
\fancyhead[L]{\nouppercase{\leftmark}}
\fancyhead[R]{}

\usepackage[T1]{fontenc}
\usepackage{titlesec, blindtext, color}
\definecolor{gray75}{gray}{0.75}
\newcommand{\hsp}{\hspace{10pt}}
\titleformat{\chapter}[hang]{\Huge\bfseries\raggedright}{\thechapter\hsp\textcolor{gray75}{|}\hsp}{0pt}{\Huge\bfseries}


%%%%%%%%%%%%%%%%%%%%%%%%%%%%%%%%%%%%%%%%%%%%%%%%%%%%%%%%%%%%%%%%%%%%%
%% If issues arise when submitting your manuscript, you may want to
%% un-comment the next line.  This provides information on the
%% version of every file you have used.
%%%%%%%%%%%%%%%%%%%%%%%%%%%%%%%%%%%%%%%%%%%%%%%%%%%%%%%%%%%%%%%%%%%%%
%%\listfiles

%%%%%%%%%%%%%%%%%%%%%%%%%%%%%%%%%%%%%%%%%%%%%%%%%%%%%%%%%%%%%%%%%%%%%
%% Place any additional macros here.  Please use \newcommand* where
%% possible, and avoid layout-changing macros (which are not used
%% when typesetting).
%%%%%%%%%%%%%%%%%%%%%%%%%%%%%%%%%%%%%%%%%%%%%%%%%%%%%%%%%%%%%%%%%%%%%
\newcommand*\mycommand[1]{\texttt{\emph{#1}}}

\DeclareAcronym{PDB}{
    short = PDB,
    long = Protein Data Bank
}

\DeclareAcronym{GAG}{
    short = GAG ,
    long = glycosaminoglycan
}


\newcommand{\squeezeup}{\vspace{-2.5mm}}


\begin{document}

\begin{titlepage}
\begin{center}

\vspace*{3cm}
\renewcommand{\thefootnote}{\fnsymbol{footnote}}
\pagenumbering{gobble}

\includegraphics[]{pictures/PNGLogo.png}
\vspace{0.2cm}


\vspace{2cm}
\textbf{\LARGE{Development of Computational Tools for the Rational Design of Glycosaminoglycan Mimetics}} \\

by Eric David Boittier \\

\vspace{0.5cm}
\\
Supervisor: Associate Professor Vito Ferro
\\
Co-supervisor: Doctor Neha S. Gandhi
\\
\vspace{3cm}
\textit{A	Research report submitted for	the	degree	of	Bachelor of	Advanced Science	(Honours)	at
The	University	of	Queensland	in	May	2018}
\\School	of	Chemistry	and	Molecular	Biosciences
\end{center}
\end{titlepage}


\newpage

% \addtolength{\topmargin}{-3in}
% \addtolength{\textheight}{-3in}

% 	\addtolength{\textwidth}{1in}

% 	\addtolength{\topmargin}{-.5in}
% 	\addtolength{\textheight}{1.75in}
	


\begin{footnotesize}

{\noindent
\textbf{Declaration by the author}\\
This	research	report	is	composed	of	my	original	work,	and	contains	no	material	
previously	published	or	written	by	another	person	except	where	due	reference	has	been	made	in	the	text.	I	have	clearly	stated	the	contribution	by	others	to	jointly-authored	works	that	I	have	
included	in	my	report.}\\
\indent I	have	clearly	stated	the	contribution	of	others	to	my	research	report	as	a	whole,	
including	statistical	assistance,	survey	design,	data	analysis,	significant	technical	procedures,	professional	editorial	advice,	and	any other	original	research	work	used	or	reported	in	my	
report.	The	content	of	my	report	is	the	result	of	work	I	have	carried	out	since	the	
commencement	of	my	honours	research	project	and	does	not	include	a	substantial	part	of	work	
that	has	been	submitted to	qualify	for	the	award	of	any	other	degree	or	diploma	in	any	
university	or	other	tertiary	institution.	I	have	clearly	stated	which	parts	of	my	research	report,	if	
any,	have	been	submitted	to	qualify	for	another	award.\\
\indent I	acknowledge	that	copyright	of	all	material	contained	in	my	research	report	resides	with	
the	copyright	holder(s)	of	that	material.
\\
\\
\textbf{Statement	of	Contributions	to	Jointly	Authored Works	Contained	in	the	Research	Report}
\\
No	jointly-authored	works.
\\
\\
\textbf{Statement	of	Contributions	by	Others	to	the	Research	Report	as	a	Whole}
\\
No	contributions	by	others.	or
List	the	significant	and	substantial	inputs	made	by	others	to	the	research,	work	and	writing	
represented	and/or	reported	in	the	research	report.	These	would	include	significant	
contributions	to:	the	conception	and	design	of	the	project;	non-routine	technical	work;	analysis	
and	interpretation	of	research	data;	drafting	significant	parts	of	the	work	or	critically	revising	it	
so	as	to	contribute	to	the	interpretation.
\\
\\
\textbf{Published	Works	by	the	Author	Incorporated	into	the	Research}\\
None.	or
List	the	publications	using	the	standard	citation	format	for	your	discipline	followed	by	a	short	
statement	about	where	those	published	works	have	been	incorporated	into	the	research	report
proposal.	For	example:
Citation	format	for	paper	– Incorporated	as	Chapter	3.
Citation	format	for	paper – Partially	incorporated	as	paragraphs	in	Chapters	2	and	5.
\\
\\
\textbf{Additional	Published	Works	by	the	Author	Relevant to	the	Research	Report	or	but	not	Forming Part of it}\\	
None.	or
List	the	publications	using	the	standard	citation	format	for	your	discipline.	The	intention	is	give	
your	research	report	assessors	a	summary	of	all	the	academic	writing	that	has	come	out	of	your	
project,	not	to	provide	a	bibliography	of	all	your	published	work.	
\\
\\

\textbf{Signature	of	Author: \rule{4cm}{0.4pt}       Date: \rule{4cm}{0.4pt}}
\\
\\
\textbf{Principal	Supervisor	Agreement}
\\
I	have	read	the	final	report	and	agree	with	the	student’s	declaration.
\\
\\
\textbf{Signature	of	Principal	Supervisor: \rule{4cm}{0.4pt}        Date: \rule{4cm}{0.4pt}}

\end{footnotesize}
\newpage
\pagenumbering{roman} 
\chapter*{Acknowledgements}
\addcontentsline{toc}{chapter}{Acknowledgements}

\newpage

%%%%%%%%%%%%%%%%%%%%%%%%%%%%%%%%%%%%%%%%%%%%%%%%%%%%%%%%%%%%%%%%%%%%%
%% The manuscript does not need to include \maketitle, which is
%% executed automatically.  The document should begin with an
%% abstract, if appropriate.  If one is given and should not be, the
%% contents will be gobbled.
%%%%%%%%%%%%%%%%%%%%%%%%%%%%%%%%%%%%%%%%%%%%%%%%%%%%%%%%%%%%%%%%%%%%%



{\setstretch{2} 
\renewcommand{\contentsname}{Table of Contents}
% \renewcommand{\thesubfigure}{\Alph{subfigure}}

\tableofcontents
\listoffigures
\addcontentsline{toc}{chapter}{List of Figures}
\listoftables
\addcontentsline{toc}{chapter}{List of Tables}
\chapter*{Acronyms}
\addcontentsline{toc}{chapter}{Acronyms}
\begin{multicols}{2}
\setstretch{1}
\DeclareInstance{acro-title}{empty}{sectioning}{name-format =}
\printacronyms[heading=none]

\end{multicols}
}


%%%%%%%%%%%%%%%%%%%%%%%%%%%%%%%%%%%%%%%%%%%%%%%%%%%%%%%%%%%%%%%%%%%%%
%% Start the main part of the manuscript here.
%%%%%%%%%%%%%%%%%%%%%%%%%%%%%%%%%%%%%%%%%%%%%%%%%%%%%%%%%%%%%%%%%%%%%
\pagebreak
\pagenumbering{arabic} 

\chapter*{Abstract}
\addcontentsline{toc}{chapter}{Abstract}

\newpage

\chapter{Computer Aided Drug Design \& Glycosaminoglycans}
\section{Adrift in Chemical Space}

\newpage
\section{Research Aims}
{\setstretch{1.8}
The aim of this project was to increase the accuracy of \textit{Vina-Carb} when predicting the structure of \acs{GAG}--Protein complexes by optimizing and developing new and existing scoring functions.
Alongside this aim, analyses of crystallographic data of \acs{GAG}--Protein complexes available in the \ac{PDB} were undertaken. 
Towards these goals, bespoke analytical tools necessary for this task were developed from new and existing software. This code was made entirely open-source. 
Tools from this code base considered particularly serviceable to the computational glycoscience community were hosted online as the Web-accessible server, \textit{GlycaTorch}, which aims to `shine a light on carbohydrate--protein interactions'.
}

\chapter{Methods Development in Protein-Ligand Docking}

\chapter{Data Mining the Protein Data Bank for Glycosaminoglycan Interactions}
\section{Abstract}

\section{Motivation}

\section{Ring Conformations and Glycosidic Torsions}
\subsection{Methods}

\subsection{Results and Discussion}

\section{Intermolecular Interactions}
\subsection{Methods}

\subsection{Results and Discussion}

\chapter{Development of a Glycosaminoglycan Specific Scoring Function}
\section{Abstract}

\section{Motivation}

\section{Glycosidic Torsion Energy Penalty Functions}
\subsection{Methods}
\subsection{Results and Discussion}

\chapter{Benchmarking Docking Software for Glycosaminoglycans}
\section{Abstract}
\section{Methods}
\section{Results and Discussion}

\chapter{GlycaTorch Web Server}



\newpage
{\setstretch{-1}
\bibliography{ref.bib}
}

%%%% un-comment for appendix
% \newpage
% \section{Appendix}


\end{document}

